\chapter{Material complementar}
\label{ch:complementar}

\section{Registros em vídeo}
\label{sec:online}



%
%%\subsection{}
%%%
%%``Arte educador, artista plástico, músico, clown e luthier experimental"\footnote{Disponível em: \url{http://rodrigooliverio.blogspot.com.br/p/info.html}. Acesso em 7 de março de 2017.} Atua como \textit{Luthier} no sentido convencional, e também com luteria experimental. Participa do grupo GEM.
%
%% 
%
%
%%\subsection{Fernando Torres}
%%\label{subsec:ftorres}
%%Fernando Torres é o principal responsável pelo espaço Plano B\footnote{Ver \ref{subsec:planob}}
%%fstorress@gmail.com
%
%%\subsection{Thelmo Christovam}
%%\label{subsec:tchristovam}
%%
%
%%Atualmente está envolvido com o projeto de um saxofone modificado semelhante ao Weevil Saxophone, desenvolvido pela empresa britânica Bugbrand junto ao saxofonista israelita-sueco Dror Feiler\footnote{Disponível em: \url{http://www.bugbrand.co.uk/bugbrand_old/pages/weevilsaxes.htm}. Acesso em 20 de janeiro de 2017.}
%%http://vgerme.blogspot.com.br
%% https://en.wikipedia.org/wiki/Dror_Feiler
%% http://radiofreidamiao.blogspot.com.br/search?zx=aa1aadcd98e3dbf6
%
%
%%\subsection{Gregory Slivar}
%%\label{subsec:gslivar}
%
%%\subsection{Vanessa de Michelis}
%%\label{subsec:vdmichelis}
%%\url{http://humanifestation.blogspot.com.br/}
%%\url{https://vimeo.com/user1245142}
%
%%\subsection{Barulho Max}
%%\label{subsec:bmax}
%%\url{https://youtu.be/41AnYPzZxp4}
%
%%\subsection{Caleb Luporini}
%
%%\subsection{Objeto Amarelo}
%
%%\subsection{Flávia Pinheiro}
%%https://www.facebook.com/flavia.pinheiro.549
%% http://jconline.ne10.uol.com.br/canal/cultura/musica/noticia/2017/09/17/performance-artistas-exploram-possibilidades-do-corpo-com-o-som-307162.php
%% https://www.facebook.com/ContatoSonoro/
%
%%\subsection{NuSom}
%%\label{subsec:nusom}
%%Projeto Móbile
%%Núcleo de Estudos em Sonologia da USP, coordenado por Fernando Iazzetta, Rogério Costa, Marcelo Queirós e Fábio Kon.
%%%http://www2.eca.usp.br/nusom/
%
%% Como você descreveria o seu processo de construção?
%
%% Que nome você dá ao isso que você constrói?
%
%% O que você ou outras pessoas fazem com estes objetos? Como você classificaria isso?
%
%% Algum destes projetos é mais representativo do seu trabalho?
%
%% Você se considera influenciado por outros artistas nesta produção?
%
%% Que outras pessoas você conhece que desenvolvem algo semelhante?
%
%%\subsection{Roberto Michelino}
%%\label{subsec:rmichelino}
%% https://vimeo.com/136679465
%%Artista plástico
%
%%Orquestras de floppy disks, mais a ver com síntese acústica do que com manipulação de mídias.
%% https://www.pri.org/stories/2014-07-30/heres-how-old-hard-drive-can-become-musical-instrument
%% hd as speaker https://www.youtube.com/watch?v=uM63nUaGgao
%%https://m.facebook.com/story.php?story_fbid=446387432222697&id=422700704591370
%
%%https://youtu.be/Lqej96ZVoo8 Performance for Noise Intoners
%%https://youtu.be/BYPXAo1cOA4
%%http://x-traonline.org/article/no-ghost-appears-luciano-chessas-reconstructions-of-the-futurist-intonarumori/
%%http://www.idmil.org/projects/intonaspacio
%%https://en.wikipedia.org/wiki/Luciano_Chessa#cite_note-15
%
%%Michel Deneuve, intérprete do Cristal Baschet desde o final dos anos 1970
%%https://vimeo.com/36338812
%%The passage struck me as the definition of our work - to invent forms that are not only sculptures with a decorative impact, but that produce music, thus letting people express themselves whether by playing them or by taking advantage of the sculptures  011017
%%http://francois.baschet.free.fr/front.htm
%%http://www.baschet.org
%%http://frenchsculpture.org/en/artist/baschet-brothers
%%https://web.archive.org/web/20110927184645/http://www.er.uqam.ca/nobel/baschet/
%%https://en.wikipedia.org/wiki/Baschet_Brothers
%
%%Robert Lebel / Arturo Schwarz on Duchamp
%
%%Raymond Wilding-White - Book on electronic music with some of the Tudor stuff
%%Lowell Cross - Laser system
%%Bob Bielecki
%% http://www.bruceduffie.com/tudor3.html
%% definição de dicionário ?
%% \cite[p.~xvii]{kartomi1990} ``If we define the term \textit{musical instruments} in the usual (though admittedly limited) way as ``implements used to produce music, especially as distinguished from the human voice'', where \textit{music} is defined according to its specific meaning in the relevant cultural context, then a few cultures may be isolated as having no musical instruments at all. These cultures have the materials and technological skill to make instruments but for various reasons choose not to. They include the Veddas of Srilanka, the Andaman Islanders of the Bay of Bengal, the Todas of southernd India, the Yami of Botel Tobago (south of Taiwan), the Fuegians at the southern tip of South America, the Lapps, the Celtic inhabitants of the Isles of Aran (Ireland) and the Outer Hebrides (Scotland), and the now extinct full-blooded Tasmanian Aborigines of Australia (verbal communication from Gregory Hurworth)''.
%
%% https://en.wikipedia.org/wiki/Yuri_Landman
%% http://www.hypercustom.nl/
%% https://consequenceofsound.net/2011/06/audio-archaeology-yuri-landmans-hypercustom-guitars/
%
%%Exemplos David Tudor
%%https://www.youtube.com/watch?v=fNvH2fzKNMQ&feature=youtu.be - Dave Tudor
%%https://www.youtube.com/watch?v=kijFn_yxiLk
%%https://www.youtube.com/watch?v=RhHkM5vL6Dc
%
%
%%https://www.youtube.com/watch?v=GLZOxwmcFVo
%

%%\begin{description}[noitemsep]
%%\item[Nome] Henrique Iwao Jardim da Silveira
%%\item[Pseudônimos]
%%\item[Data de nascimento]
%%\item[Local de atuação principal] São Paulo - SP, Belo Horizonte - MG
%%\item[Formação]
%%\item[Local de origem]
%%\item[Associações]
%%\item[Descrição] 
%%\item[Projetos] 
%%\begin{description}
%%\item
%%\item[xxxx] 
%%\item[xxxx]
%%\end{description}
%%\item[Página da internet]
%%\end{description}
%
%
%%Material Ale
%%Henrique Iwao
%%Sítio Internet: http://henriqueiwao.blogspot.com.br
%%Henrique Iwao, artista brasileiro, é um dos coordenadores e criadores do coletivo de música
%%experimental Ibrasotope (http://ibrasotope.com.br/). Em seu blog é possível ver alguns de seus
%%trabalhos com circuito alterado, como é o caso do Atari Punk Console de Parede
%%(http://henriqueiwao.blogspot.com.br/2011/02/atari-punk-console-de-parede.html), circuito
%%montado utilizando como base a parede do espaço do Ibrasotope, onde ocorria uma oficina de
%%Pan\&Tone. Outros posts com um uso no mínimo inusitado do circuito alterado são Avenida Paulista
%%Eu Te Amo no. 1 (http://henriqueiwao.blogspot.com.br/2008/09/avenida-paulista-eu-te-amo-no-
%%1.html) e 
%% http://henriqueiwao.seminalrecords.org/curriculo/dossier-iwao/
%% http://www.uiadiario.com.br/evento/henrique-iwao-marcelo-muniz-em-radio-radio-e-mini-tabuas-miscelania/
%%http://henriqueiwao.seminalrecords.org/avenida-paulista-eu-te-amo-no-1/
%%http://henriqueiwao.seminalrecords.org/avenida-paulista-eu-te-amo-no-2/ https://www.flickr.com/photos/henriqueiwao/3881931744/in/set-72157615813362191/
%
%%Perguntas:
%
%%- Como você ficou sabendo da existência da Gatorra e do Tony? Como você entrou em contato com a Gatorra?
%%- O que motivou o seu interesse por encomendar uma Gatorra?
%%- Como você usa o instrumento?
%%- Quais os pontos preferidos e quais os problemas que você encontrou no uso da Gatorra?
%%- Como você toca a Gatorra? Você já usou a Gatorra em alguma gravação? Você poderia mostrar exemplos do uso da Gatorra?
%
%%- How did you learn about the Gatorra? How did you come across it?
%%- How do you use it?
%%- What are your favourite features in the Gatorra, and what are its main shortcomings, in your opinion?
%%- How do you play it? Have you ever recorded it? Could you show some examples of your Gatorra playing?
%
%%http://enciclopedia.itaucultural.org.br/pessoa420662/tato-taborda
%%http://culturadigital.br/forum2010/2010/10/29/dona-geralda-lhe-espera-no-forum-da-cultura-digital-2010/
%%http://lattes.cnpq.br/5444189855425611
%%https://www.flickr.com/photos/flimultimidia/5183601536
%%https://www.flickr.com/photos/flimultimidia/5180978034/sizes/l/
%%http://radios.ebc.com.br/semana-tematica-mec-fm/edicao/2015-08/koellreutter-100-anos-tato-taborda-narra-trajetoria-do-musicologo
%%http://www.overmundo.com.br/agenda/multiplicidade-06-colecao-de-inutilezas-com-tato-taborda-geralda
%%http://www.jb.com.br/cultura/noticias/2007/07/16/tato-taborda-toca-no-multiplicidade/
%
%%http://tvbrasil.ebc.com.br/estudiomovel/episodio/musica-ilustracao-e-projetos-culturais
%
%%Felisberto - Radiopiano - Piano preparado mais rádios acionados por teclas específicas.
%
%%Câmara anecóica do Inmetro em Xerém
%
%
%% Vídeos
%% https://youtu.be/EHPxQPDzXZg
%% https://youtu.be/wvDnbOa1obU
%
%%Como a relação com os materiais se modifica durante o processo de racionalização da música? É possível isto ser discutido? O quanto este aspecto de relaxamento observável no contexto musical do séc. XX pode ser relacionado com seus meios?
%%O quanto a relação entre produção musical e instrumentos se modifica? Existem registros sobre o quão técnica era a abordagem antiga.
%
%%[HT p 27] ``choques" de Benjamin - o impacto dos processos industriais e tecnológicos modernos no indivíduo. Correspondência entre a recepção distraída e tecnológica e a reprodutibilidade técnica. ``[A] reprodução tecnológica, ao contrário, envolve uma forma de memória artificial ou tecnológica, cujas imagens são meras cópias, imitações, citações". [ht p 28]
%
%%Yet, to reduce the processes of techno-culture to a matter of capitalist instrumentality, to a matter of accumulation strategies and stratification patterns, is to do exactly what capitalism itself does. It is to try to resecure or reinstrumentalize, to control, the unsettling processes of techno-culture.[ht 154]
%
%%\subsection{Al Revés}
%%\label{subsec:alreves}
%%Al Revés é um coletivo baseado em São Paulo, cujos participantes atualmente são Bruno Hiss e Alexandre Marino.
%
%%\subsection{Natacha Maurer}
%%\label{subsec:nmaurer}
%%Produtora de eventos e também curadora do Ibrasotope, FIME e Ciclo de Música Experimental\footnote{Ver \ref{subsec:ibrasotope}, \ref{subsec:fime} e \ref{subsec:cme}}. Integrante do Brechó de Hostilidades Sonoras junto com Marcelo Muniz\footnote{Ver \ref{subsec:brecho} e \ref{subsec:mmuniz}}
%% http://natmaurer.tumblr.com
%
%%\subsection{Orquestra Organismo}
%%\label{subsec:orquestraorganismo}
%%Orquestra Organismo
%%Sítio Internet: http://organismo.art.br/
%%Material Alê
%%Coletivo brasileiro, define-se como “fluxo colaborativo e interdisciplinar que se manifesta
%%através de ações diretas e poéticas.” A atuação do grupo procura abordar “questões relacionadas a
%%agenciamento, ritualização e formação de circuitos. Se dispõe a proporcionar encontros relacionais
%%não-hierárquicos com diversos organismos coletivos, instituições e demais interessados.”
%%Ressaltando grande parte das questões discutidas nesta dissertação. Dentro de suas atuações,
%%encontra-se o conceito de “Artesanato de Volts”, boa forma em português para descrever o tipo de
%%ação envolvida na metodologia do Circuito Alterado, segundo sua descrição, o termo é utilizado
%%para:
%%definir uma busca de produção artística e poética que utiliza como matériaprima
%%eletrônica de baixo custo e quando possível reciclada. A ênfase também
%%é por uma pesquisa com linguagens e métodos computacionais para uso
%%criativo, como construção de instrumentos musicais, audiovisuais ou mesmo
%%invenção de novas interfaces e suportes para a expressão de ideias.
%
%%\subsection{Bugigangsters}
%%\label{subsec:bugigangsters}
%

%
%%

%% http://cargocollective.com/thiagosalas/MNEMORFOSES
%% https://www.google.com.br/search?client=opera&q=Thiago+Salas&sourceid=opera&ie=UTF-8&oe=UTF-8
%
%%http://www.sescsp.org.br/programacao/113816_PERCUSSAO+NA+SUCATA
%%https://www.facebook.com/pg/LoopB/about/?ref=page_internal
%%http://demo-tapes-brasil.blogspot.com.br/2011/12/loop-b-techno-totem-1995.html
%% https://demo-tapes-brasil.blogspot.com.br/2013/04/loop-b-techno-industrial-1997.html?m=1
%
%% The brazilian Mark E. Smith - https://pitchfork.com/reviews/albums/14489-the-terror-of-cosmic-loneliness/
%
%% https://musicalamizade.wordpress.com/tag/marcelo-birck/
%% Diego da GAtorra: https://m.youtube.com/watch?v=hXWjil4mM8c
%% http://odiluvio.blogspot.com.br/2014/02/noentrevista-tony-da-gatorra.html?m=1
%
% %``É só pessoa mesmo que tem grana, que tem dinheiro pra gastar com esse instrumento, é só pessoa intelectual, colecionador de instrumento, é gringo, é músico gringo que vem aqui pro Brasil [...] e sabe que aqui no Brasil tem outros instrumentos mais chiquezinhos que berimbau''
%
%
%%%https://vimeo.com/13232951 Eu não sou músico
%%
%%%``Não mudou nada, eu continuo pobre''\footnote{Tony da Gatorra em entrevista ao programa \textbf{Singulares}, da TVE-RS, exibido originalmente em 04 de abril de 2016. Disponível em: \url{https://www.youtube.com/watch?v=WS9EWO03k1k}. Acesso em 14 de maio de 2018}.
%%
%%% Vídeo dissenso - construção 95 / 96
%
%
%%http://marcelobirck.blogspot.com.br/2008/07/tony-da-gatorra.html
%
%%https://musicalamizade.wordpress.com/2009/04/27/coletivo-concerto-grosso-na-torre-da-hidraulica/
%
%% http://odiluvio.blogspot.com.br/2014/02/noentrevista-tony-da-gatorra.html
%
%% Tony dissenso http://tonydagatorra.blogspot.com.br/2011/09/26092011.html
%% http://artesemlei.blogspot.com.br/2011/08/3-dis-experimental.html
%% https://www.youtube.com/watch?v=iRDWoOjMxSM
%
%%http://www.nme.com/news/super-furry-animals/50417
%%Guitorra : http://tonydagatorra.blogspot.com.br/2007/01/31012007.html?m=1
%%Virou gatorra 11 ou 12
%
%%Perspectiva do Circuit Bending
%%https://web.archive.org/web/20070101000000*/http://www.panetone.org
%%https://web.archive.org/web/*/www.panetone.net
%%https://vimeo.com/cristianorosa
%%https://foxtonemusic.com/shop/stand-alone/sismo-twin-t-bass/
%%https://youtu.be/r6EvqhTc1X0
%%https://youtu.be/VswFYf_fPTg
%%https://archive.org/details/panetone
%%http://www.vice.com/pt_br/read/tcp-os-ruidos-do-corpo
%%http://getlofi.com/panetone-is-from-brazil/
%%https://youtu.be/23LWN7TVSBw
%%https://youtu.be/rwRTtCOpXI4 Matéria Circuit Bending
%
%%Pan\&Tone (aka Cristiano Rosa)
%%Sítios internet: http://panetone.net e http://labelcreativa.org
