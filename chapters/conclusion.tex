% The \phantomsection command is needed to create a link to a place in the document that is not a
% figure, equation, table, section, subsection, chapter, etc.
%
% When do I need to invoke \phantomsection?
% https://tex.stackexchange.com/questions/44088/when-do-i-need-to-invoke-phantomsection
\phantomsection

% ---
% Considerações Finais (outro exemplo de capítulo sem numeração e presente no sumário)
% ---
\chapter*[]{\lang{Final Remarks}{Considerações Finais}}
\phantomsection
\addcontentsline{toc}{chapter}{Considerações Finais}%



\section{Porque fazer?}
\begin{citacao}
De volta à metade dos anos 1960, o mcluhanismo fora inventado
como o credo do Centro Vital. Duas décadas depois, o signifi cado dessa teoria essencial no meio da elite dos Estados Unidos moveu-se para a direita. Com a Esquerda da Guerra Fria desacreditada, muitos de seus membros acharam consolo ideológico no renascimento do liberalismo de livre mercado nos anos 1970: o neoliberalismo.\cite[347]{Barbrook2009}
\end{citacao}

\begin{citacao}
Dos sistemas de câmeras de vigilância aos programas de monitoramento de mensagens eletrônicas, o governo dos Estados Unidos e seus aliados sistematicamente adquiriam as ferramentas para uma vigilância constante de toda a população global. No setor privado, as tecnologias da informação similarmente revitalizaram as hierarquias tayloristas. (...) graças ao panóptico em rede, a elite corporativa era agora capaz de controlar suas vidas muito mais detalhadamente do que no passado fordista. O tecno-coletivismo do mcluhanismo metamorfoseou-se no tecno- autoritarismo da consultoria gerencial de McKinsey. \cite[345]{Barbrook2009}
\end{citacao}

``No momento em que todos tivessem acesso à Internet, a democracia participativa e a criatividade cooperativa seriam a ordem do dia. Entretanto'' \cite[360]{Barbrook2009}

\begin{citacao}
Todos os sonhos de democracia participativa e criatividade cooperativa seriam realizados dentro da aldeia global por vir. Em estágios iniciais da modernidade, esses princípios libertários foram somente parcialmente realizados. Felizmente, uma vez que estivessem conectados à Internet, todos – inclusive os descendentes dos escravos – desfrutariam dos benefícios da democracia da alta tecnologia jeffersoniana.
\end{citacao}
  
 \begin{citacao}
 
   A partir das relações do homem com a realidade, resultantes de estar com ela e de estar nela, pelos atos de criação, recriação e decisão, vai ele dinamizando o seu mundo. Vai dominando a realidade. Vai humanizando-a. Vai acrescentando a ela algo de que ele mesmo é o fazedor. Vai temporalizando os espaços geográficos. Faz Cultura. E é ainda o jogo destas relações do homem com o mundo e do mundo e do homem com os homens, desafiado e respondendo ao desafio, alterando, criando, que não permite a imobilidade, a não ser em termos de relativa preponderância, nem das sociedades nem das culturas. E, à medida que cria, recia e decide, vão se conformando as épocas históricas. É também criando, recriando e decidindo o que o homem deve participar destas épocas. \cite[60]{Freire2015}
    \end{citacao} 

   \begin{citacao}
   Uma das grandes, senão a maior, tragédia do homem moderno está em que é hoje dominado pela força dos mitos e comandado pela publicidade organizada, ideológica ou não, e por isso vem renunciando cada vez, sem o saber, à sua capacidade de decidir. Vem sendo expluso da órbita das decisões. As tarefas do seu tempo não são capatadas pelo homem simples, mas a eles apresentadas por uma ``elite" que as interpreta e as entrega em forma de receita, de prescrição a ser seguida. E, quando julga que se salva seguindo as prescrições, afoga-se no anonimato nivelador da massificação, sem esperança e sem fé, domesticado e acomodado: já não é sujeito. Rebaixa-se a puro objeto. Coisifica-se. \cite[60]{Freire2015}
 \end{citacao} 

As eleições de 2018 provaram o potencial destruidor dos novos meios de comunicação, esse sujeito objeto massificado, impulsionado pela era da pós-verdade, em ambientes completamente controlados por algoritmos que não se sabe o que e quem controlam. O projeto Banda Aberta foi uma tentativa de dialogar com isso, propor novas formas de interação, mas a relação de controle imposta pela separação condutor/audiência, compositor intérprete não me deixou ainda confortável. 

\begin{citacao}
A música não pode ser uma linguagem nem fixada, nem meramente codificada pelo uso. A música faz-se e inventa-se constantemente, procura-se um sentido, e qualquer passagem misteriosa e singular — na verdade bastante singular — entre natureza e
cultura. \cite{Schaeffer2007}

Decidir seguir a carreira de professora, de ter um compromisso com a educação e as potências que emanar dessas relações, faz pensar em ferramentas que possam ser apreendidas de uma maneira mais abrangente. Tive felizmente, no final deste processo a oportunidade de lecionar no recém criado cuso de Som e Imagem em Movimento da Universidade Federal do Sul da Bahia, na disciplina de Capatação Edição e Produção Sonora, em parceria com o professor Leonardo Souza, do curso de Artes do Corpo em cena, que foi uma oportunidade de utilizar o Playsound na prática em um contexto educacional.

Pudemos usar a ferramenta em diverentes contextos educacionais. Numa aula sobre timbre, por exemplo, reunimos num link vários exemplos de uma mesma nota gerada por instrumentos diferentes como diapasão, violino, clarinete, oboé, trompete, flauta, voz, piano, guitarra elelétrica e diversos modelos diferentes de sintetizadores facilmente a partir do acervo do Freesound. Pudemos tocar todos esses sons facilmente e compará-los inclusive analisando os diferentes perfis de espectro de frequências sonoras.

\footnote{\url{http://www.playsound.space/sounds=308270,308618,309333,290401,43461,314864,399466,295858,278084,334534,428800,246658,357370,355118,356661,374567,220747}}









\subsection{O que fazer?} 

\todo[inline]{posicionar essa citacao}
\begin{citacao}
Tanto multimídia como intermídia, são categorias interdisciplinares que, como colagem ou síntese-qualitativa, colocam em questão as formas de produção-criação individual e sobretudo a noção de autor. A criação é hoje o resultado da interação dessas práticas, como forma de tradução e inter-relação. O que não quer dizer que já não seja possível instaurar um estilo: ele é hoje a marca invariante que estabelece a diferença transmutadora em quaisquer dos suportes utilizados. O diálogo entre o singular-individual (ego) e o coletivo (superego) é uma das caracterísiticas da prática tecnológica. Por outro lado, os meios tecnológicos absorvem e incorporam os mais diferentes sistemas sígnicos, traduzindo as diferentes linguagens históricas para o novo suporte. Essas linguagens transcodificadas efetivam a colaboração entre os diversos sentidos, possibilitando o trânsito intersemiótico e criativo entre o visual, o verbal, o acústico e o tátil \cite[66]{JulioPlaza1969}
\end{citacao}

Os projetos desenvolvidos no decorrer desta tese também apontam para uma série de desejos de desenvolvimentos futuros. 

- ferramenta colaborativa
- tracker
- sons autorais
- remixagem com faixas 
- sintetizador
- radio




\end{citacao}

