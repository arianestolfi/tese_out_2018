{\chapter[Recursos]{Recursos}}
\label{ch:recursos}
% ---


% Multiple-language document - babel - selectlanguage vs begin/end{otherlanguage}
% https://tex.stackexchange.com/questions/36526/multiple-language-document-babel-selectlanguage-vs-begin-endotherlanguage

 



% As this page is not being completely filled, it is generating the page bottom bad box.
% Fix Underfull \vbox (badness 10000) has occurred while \output is active
%
% \flushbottom vs \raggedbottom
% https://tex.stackexchange.com/questions/65355/flushbottom-vs-raggedbottom

Aqui reunimos informações sobre alguns recursos que foram úteis no desenvolvimento dessa pesquisa. 

\section{Linguagens}
\subsection{HTML5}
HTML 5 é a última versão da linguagem HTML. HTML ou Hypertext Markup Language é a linguagem base por trás da internet. Funciona através de \emph{tags}, que são palavras escrita dentro de marcas especiais que determinam comportamentos de partes específicas de um texto. Através de marcas especiais, a linguagem caracteriza partes de um texto de acordo com sua função sintática ou estética. Nessa nova versão, a linguagem dá suporte para tocar arquivos de vídeo ou de áudio diretamente do navegador. Abaixo, um exemplo de arquivo HTML com algumas \emph{tags} básicas.

\begin{verbatim}
<html>
<head>
    <title>O título da página</title>
    <script>
    //É possível colocar scripts aqui
    </script>

</head>

<body>

    <h1>Isso é um cabeçalho</h1>
    <p>Isso é um parágrafo</p>

<audio id="som">
    <source src="video.mp4" type="video/mp4">
    <source src="video.ogg" type="video/ogg">
</audio>

</body>
</html>
\end{verbatim}

\subsection{JavaScript}
JavaScript é uma linguagem de programação voltada para programação de aplicativos na web. Tem como característica principal o fato de não ser compilada, o que a coloca dentro da categoria de linguages de ``script'', que são interpretadas sem a necessidade de fechamento. Isso colabora também na difusão da linguagem, já que os exemplos que estão \emph{online} podem ser acessados por qualquer um através dos inspetores de código do navegador. Abaixo, um exemplo simples de como tocar um som em JavaScript: 

\begin{verbatim}
function toca(som) {
    document.getElementById(som).currentTime = 0;
    document.getElementById(som).play();
}
\end{verbatim}

\subsubsection{Node.js}
Uma das tecnologias que utilizamos no Playsound para fazer a parte de processamento no servidor, foi Node.js\footnote{\url{https://nodejs.org/en/}}. Node é uma tecnologia que tem estado em evidência nos últimos anos por oferecer um servidor programado em JavaScript. Isso facilita a utilização modular de uma série de pacotes pré-programados que são instalados através de um gerenciador de pacotes. Node facilita a utilização de técnicas de desenvolvimento assíncronas, onde as páginas são alteradas sem a necessidade de recarregamento. 

%\subsubsection{Angular}
%Angular.js é um 

\subsubsection{JSON}
JSON ou JavaScript Object Notation\footnote{https://www.json.org/} é um formato para registro de dados que pode ser lido por humanos ou por máquinas. É organizado em objetos que contém séries de pares de nomes e valores de variáveis, que são determinadas pelos programadores. JSON é uma das tecnologias por trás dos bancos de dados mais modernos e da maioria das ferramentas que disponibilizam acesso a dados por REST API's.


%\subsection{CSS}

\subsection{API's}
\begin{citacao}
Massive amounts of digital data can now be researched, collected, interpreted, reformatted, and displayed for the purpose of art-making. This gives data a chance to be reborn toward aesthetic, communicative, or social purposes. Perhaps the simplest idea of this new art is the idea of copy and paste, allowing digitalized data to move from one location to another. From this core idea the rise of an internet culture, and network capabilities expands this to global dissemination of content. From here, the dynamics of this network culture permits artifacts to become art systems. All these aspects are dependent on the technological capacities. The technologies support these three aspects: cut/paste, networking and dissemination, and artifact/systems while simultaneously advancing the ease by which they can be performed. In this manner the collaboration is growing in both the number of participants as users and the number of participants as creators. Still, due to human practice and change through learning, our relationship to technology is always in fiery negotiation. The public Interface can be regarded as a technological construct as well as a cultural artifact as the elements in cyberspace (such as the dialogue and logic/ language patterns) become revealed via the interface. The art-making public interface is both the media and the message composited, it allows for sharing and repurposing. In this respect it fosters its own cultural artifact—artmaking that can be examined in its own right. The importance of this collective must be acknowledged as a heretofore unknown thing; this public interface has lead to a new art system. This is the foundation of a network aesthetic that will continue to evolve.
\cite[5]{Soon2011}
\end{citacao}

\subsubsection{Web Audio API}
A Web Audio API oferece um sistema de controle de processos sonoros através dos navegadores. O Design da API é determinado pelo W3C e sua implementação é feita pelas companhias que desenvolvem os navegadores, assim como as demais linguagens web. A API define um contexto de áudio (audio context) que permite o roteamento modular de cadeias de processamento de som através de JavaScript. Nesse projeto, utilizamos a API no projeto Banda Aberta e no Playsound. 

\subsubsection{Freesound API}
O site Freesound.org \cite{Font2013} é uma ferramenta bastante reconhecida para a comunidade de produtores musicais. O site, que foi crido em 2005 conta com um acervo de centenas de milhares de sons originais licenciados em algum dos tipos de licensa Creative Commons. 

Em 2011 o site foi reformulado para oferecer uma interface de acesso a seu banco de dados (restfull API) \cite{Akkermans2011} via JSON. Para acessar seu banco de dados é preciso fazer um processo de autenticação que exige uma chave (API KEY), que pode ser obtida no site. A chave padrão ofere certas restrições como limites de acesso por hora. 

A interface permite a pesquisa no banco de dados a partir de buscas textuais, ou atráves de buscas por recursos sonoros. Isso porque todo o banco de dados do Freesound passa por processos de análise computacionais, através do software Essentia \cite{Bogdanov2013}, de onde se extraem uma série de informações sobre parâmetros sonoros. Até o momento, estamos usando nessa pesquisa principalmente a busca textual, mas no futuro gostaríamos de aperfeiçoar o sistema de busca para incluir parâmetros mais complexos oferecidos pela API.

O processamento via Essencia oferece uma série de descritores espectrais, como: energia de cada faixa de frequência; complexidade espetral; contraste de espectro; dissonância; distribuição espectral; momentos centrais de banda de espectro; tom; energia por banda; saliência tonal; taxa de siêncio; entropia espectral entre outros.

Conta com descritores rítmicos, como: BPM; BPM da primeira batida; intervalos de BPM (bpm\_intervals) (first\_peak\_bpm); tempo até o início do som (onsetTimes); volume das batidas (beats\_loudness); duração da primeira batida (first\_peak\_spread). Provém também descritos tonais como: entropia; escala de acordes; acordes; progressões de acordes; afinação; contagem de picos; decaimento temporal; inaharmonicidade; entre vários descritores de áudio complexos.

Buscas pela interface de análise podem ser feitas por um acesso HTTP do tipo:

\begin{verbatim}
\url{curl https://freesound.org/api/sounds/<sound_id>
/analysis/lowlevel/pitch
\end{verbatim}

A busca textual no banco de dados do Freesound pode conter uma série de filtros como:

id  
username:   
created (date)
original\_filename
description
tag
license
is\_remix
was\_remixed
pack
pack\_tokenized
is\_geotagged
type (“``wav”'', ``aif'', ``ogg'', ``mp3'', ``m4a'' ou ``flac'')
duration:   
bitdepth:   
bitrate:    
samplerate: 
filesize:   
channels:   integer, number of channels in sound (mostly 1 or 2)
md5:    string, 32-byte md5 hash of file
num\_downloads: 
avg\_rating:     average rating, from 0 to 5
num\_ratings: number of ratings
comment:    
comments:   number of comments

\begin{verbatim}
HTTP 200 OK
Allow: GET, HEAD, OPTIONS
Content-Type: application/json
Vary: Accept

{
    "count": 1080,
    "next": "http://freesound.org/apiv2/search/text/
        ?&query=%22bass%20drum%22%20-double&page=2",
    "results": [
        {
            "id": 389153,
            "name": "but there is a bass drum",
            "tags": [
                "analog",
                "electronic",
                "electro",
                "synth",
                "sample-and-hold",
                "eurorack",
                "synthesizer",
                "bass-drum",
                "bass",
                "drum",
                "noise"
            ],
            "license": "http://creativecommons.org/
                publicdomain/zero/1.0/",
            "username": "gis_sweden"
        },
        
    "previous": null
}

\end{verbatim}



\section{Ferramentas}
\subsection{Terminal}

\begin{citacao}
Linha de comando eh o maior maior barato
Você nunca mais vai esquecer
Dar um cat no arquivo
Ls pra listar
E pra mudar o diretório eh o cd (Articuladores, )
\end{citacao}

Operar através de linha de comando, é um dos primeiros aprendizados do hacker. Se estéticamente parece uma coisa obscura, misteriosa para usuário superficial de computadores, é uma tecnologia importante para vários processos de desenvolvimento de software, incluindo também para o controle de servidores, e gerenciamento de repositórios de código. A linha de comando, que é como se chama a interface para controle dos sistemas operacionais via texto remove a interface gráfica do usuário (GUI), que contém todas as metáforas de usabilidade desenvolvidas para tornar a computação um processo mais familiar para os usuário de uma maneira geral.

Ao remover essa camada, a linha de comando exige que o operador saiba os comandos para executar as tarefas necessárias ao seu trabalho, ou pelo menos saiba como procurar como saber, mas também permite um acesso mais direto e não mediado a estruturas de dados, aplicativos e ferramentas diversas. Compreender como funciona a linha de comando é em um certo sentido também libertador, ao passo que pode conferir agilidade e poder sobre meios digitais. 

Neste trabalho, a linha de comando serviu também como inspiração para a interface de controle do projeto Banda Aberta. Nele, os comandos para se trocar som, aumentar ou diminuir o volume funcionavam através de códigos que eram colados na conversação, que apenas algumas pessoas na audiência dominavam. 


\subsection{Git}
Git é uma ferramenta de controle de versões voltadas para se criar repositórios de códigos fonte. Além do Git existem outras ferramentas similares, como CVN Com o Git, é possível que uma equipe compartilhe os mesmos arquivos, que ficam organizados em um servidor central. O git tem uma interface gráfica, mas é mais prático e eficiente de ser usado através da linha de comando. 

Quando se criamos um repositório, podemos incluir arquivos que podem ser copiados por qualquer pessoa que tiver acesso a ele. Existem repositórios provados, aos quais você em geral precisa ter que pagar, mas se você estiver disposto a disponibilizar seu código como software livre, exitem vários serviços gratuitos para hospedagem online, como Github, Bitbucket etc. 

O sistema permite que se criem galhos ou \emph{branches}, que são versões em paralelo do software em questão, e também guarda todas alterações enviadas pelos colaboradores do projeto. Funciona como sistema de backup, mas é importante principalmente para o desenvolvimento de projetos colaborativos que envolvem mais de um programador. 

Neste projeto, utilizamos o Github como plataforma para disponibilização de todos os códigos fonte desenvolvidos durante o projeto, como os do Banda Aberta, Spectrogram Player e Playsound, e também os próprios códigos fonte desta tese, que foi escrita em LaTex.

\subsection{LaTex}

Latex é uma linguagem de programação voltada para a escrita de textos científicos. Ao escrever em LaTex, a parte burocrática da escrita científica, como numerar figuras, tabelas e notas de rodapé, gerar bibliografias e posicionar figuras é feita automaticamente pelo sistema. A aparência gráfica da página é definida por um modelo, ou \emph{template}, que pode ser copiado ou fornecido por uma instituição ou comitê científico por exemplo.

A descoberta do Latex como ferramenta de produção foi de extrema importância no desenvolvimento deste projeto de pesquisa, e facilitou a produção de nove artigos que foram publicados durante esse processo. Para isso, fiz uso também da ferramenta online Overleaf, que fornece acesso online a uma ferramenta de edição colaborativa e visualização de arquivos LaTex. Para o desenvolvimento desta tese, no entanto, em um determinado momento foi necessária a migração para um ambiente de desenvolvimento local pela quantidade de arquivos que não era suportada pela versão gratuita do sistema. Além disso, com o ambiente de desenvolvimento local foi possível se desvenciliar da necessidade de conecção de internet para a escrita do documento.









