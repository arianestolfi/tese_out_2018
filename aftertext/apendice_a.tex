%
% How to fix the Underfull \vbox badness has occurred while \output is active on my memoir chapter style?
% https://tex.stackexchange.com/questions/387881/how-to-fix-the-underfull-vbox-badness-has-occurred-while-output-is-active-on-m
%

% ---

\lang
{\chapter[Appendix A]{Resources}}
{\chapter[Apêndice A]{Recursos}}
% ---


% Multiple-language document - babel - selectlanguage vs begin/end{otherlanguage}
% https://tex.stackexchange.com/questions/36526/multiple-language-document-babel-selectlanguage-vs-begin-endotherlanguage

 



% As this page is not being completely filled, it is generating the page bottom bad box.
% Fix Underfull \vbox (badness 10000) has occurred while \output is active
%
% \flushbottom vs \raggedbottom
% https://tex.stackexchange.com/questions/65355/flushbottom-vs-raggedbottom
\newpage

\section{Linguagens}
\subsection{HTML5}


\subsection{JavaScript}

\subsection{API's}
\begin{citacao}
Massive amounts of digital data can now be researched, collected, interpreted, reformatted, and displayed for the purpose of art-making. This gives data a chance to be reborn toward aesthetic, communicative, or social purposes. Perhaps the simplest idea of this new art is the idea of copy and paste, allowing digitalized data to move from one location to another. From this core idea the rise of an internet culture, and network capabilities expands this to global dissemination of content. From here, the dynamics of this network culture permits artifacts to become art systems. All these aspects are dependent on the technological capacities. The technologies support these three aspects: cut/paste, networking and dissemination, and artifact/systems while simultaneously advancing the ease by which they can be performed. In this manner the collaboration is growing in both the number of participants as users and the number of participants as creators. Still, due to human practice and change through learning, our relationship to technology is always in fiery negotiation. The public Interface can be regarded as a technological construct as well as a cultural artifact as the elements in cyberspace (such as the dialogue and logic/ language patterns) become revealed via the interface. The art-making public interface is both the media and the message composited, it allows for sharing and repurposing. In this respect it fosters its own cultural artifact—artmaking that can be examined in its own right. The importance of this collective must be acknowledged as a heretofore unknown thing; this public interface has lead to a new art system. This is the foundation of a network aesthetic that will continue to evolve.
\cite[5]{Soon2011}
\end{citacao}

\subsubsection{WebAudio API}

\subsubsection{Freesound API}

\section{Ferramentas}
\subsection{Terminal}
\subsection{Git}
\subsection{Node.js}
\subsection{Latex}

\subsection{}




\subsubsection{ }




